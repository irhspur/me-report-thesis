\documentclass{article}

\title{Project Report on Synthesizing Natural Sounding Nepali Voice}
\date{2018-04-15}
\author{Roop Shree Ratna Bajracharya}

\begin{document}
	
	\pagenumbering{gobble}
	\maketitle
	\newpage

	\tableofcontents
	\newpage

	\pagenumbering{arabic}
	
	\section{Abstract}
	Text-to-Speech (TTS) synthesis has come far from its primitive synthetic monotone voices to more natural and intelligible sounding voices. Festival is one of such systems that uses a concatenative speech synthesis method to produce natural sounding voice. This project aims to develop Nepali voice using Festival system with tools included in Festvox as a part of the Nepali-TTS project currently being conducted in Information and Language Processing Research Lab (ILPRL) in Kathmandu University. This project includes studying and improving different steps and procedures involved in the speech synthesis process and will aim to produce more natural sounding Nepali voice.
	\paragraph{}
	Keywords: Text-To-Speech, Festival Speech Synthesis, Concatenative Speech Synthesis
	\newpage
	
	\section{Introduction}
	\subsection{Text to Speech Synthesis}
	Text-To-Speech (TTS) system converts the given text to a spoken waveform through text processing and speech generation processes. These processes are connected to linguistic theory, models of speech production, and acoustic-phonetic characterization of language \cite{ReviewOfTTS:1}. There are three approaches to TTS system: 1) articulatory-model 2) parameter-based  and 3) concatenation of stored speech \cite{ReviewOfTTS:1}. This project deals with the third approach to concatenate stored speech segments (units). This approach includes storing and concatenation of speech from huge number of recorded sentences of a language to produce a speech waveform. 


	\newpage
	\bibliography{ref}
	\bibliographystyle{ieeetr}
\end{document}